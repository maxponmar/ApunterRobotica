\documentclass[10pt,a4paper]{article}
\usepackage[utf8]{inputenc}
\usepackage[english]{babel}
\usepackage{amsmath}
\usepackage{amsfonts}
\usepackage{amssymb}
\usepackage{soul}
\usepackage[margin=1in]{geometry}
\usepackage{enumitem}
\usepackage{adjustbox}

\newlength{\drop}

\usepackage{tikz}

\newcommand{\inline}[2]{%
    \begin{tikzpicture}[baseline=(word.base), txt/.style={shape=rectangle, inner sep=0pt}]% the baseline key ensures that nodes won't shift up if there's text with descenders, and the txt style removes extra spacing so you can use this inline
    \node[txt] (word) {#1};% the first argument is the contents of the main node
    \node[above] at (word.north) {\footnotesize{#2}};% the second argument is the tag; you can play with the positioning as necessary
    \end{tikzpicture}%
    }

\begin{document}

\begin{titlepage}

\drop=0.1\textheight
    \centering
    \vspace*{\baselineskip}
    \rule{\textwidth}{1.6pt}\vspace*{-\baselineskip}\vspace*{2pt}
    \rule{\textwidth}{0.6pt}\\[\baselineskip]
    {\LARGE APUNTES DE\\[0.2\baselineskip] ROBOTICA}\\[0.2\baselineskip]
    \rule{\textwidth}{0.4pt}\vspace*{-\baselineskip}\vspace{3.2pt}
    \rule{\textwidth}{1.6pt}\\[\baselineskip]
    \scshape
    MODELO CINEMATICO DE VELOCIDAD \\
    \vspace*{2\baselineskip}
    %Edited by \\[\baselineskip]
    %{\Large MAXIMILIANO PONCE\par}
    %{\itshape My notes from \\Langpill english grammar course from Udemy, and Grammarly\\\par}
    \vfill
    {\scshape Abril 2020} \\
    {\large MAXIMILIANO PONCE}\par

\end{titlepage}

\tableofcontents
\newpage

\section{Modelo cinematica de velocidad}
En esta seccion se derivan las relaciones de velocidad, con respecto a las velocidades lineales y angulares del efector final. Las relaciones de velocidad se determinan por los Jacobianos de la cinematica directa.\\

Velocidad angular: Caso de eje fijo\\
\begin{center}
	\begin{tabular}{c c}
		$\omega = \dot{\theta} \vec{k} $ & picture
	\end{tabular}
\end{center}

donde $\dot{\theta}$ es la derivada con respecto al tiempo de $\theta$, $\vec{k}$ es un vector en direccion del eje de rotacion y $\omega$ es la velocidad angular. Dada la velocidad angular del cuerpo, la velocidad lineal de cualquier punto en el cuerpo esta dada por:

$$ \upsilon = \omega \times \vec{r}$$

donde $\vec{r}$ es un vector desde el origen al punto dado. La velocidad angular $\omega$ es una propiedad de la trama anexa al cuerpo. La velocidad angular no es una propiedad de un punto particular. Asi que $\upsilon$ corresponde a la velocidad lineal de un punto mientras $\omega$ corresponde a la velocidad angular de una trama rotando.

\subsection{Metricas asimetricas}
Se dice que una matriz $S$ es asimetrica si y solo si:
$$ S^T + S = 0$$

donde,
\begin{equation}
	S =
	\begin{bmatrix}
		0    & -s_3 & s_2\\
		s_3  & 0    & -s_1\\
		-s_2 & s_1  & 0
	\end{bmatrix}
\end{equation}

\textbf{Por ejemplo}: si se definen a los tres vectores unitarios de un sistema de coordenadas como $\vec{i}$, $\vec{j}$, $\vec{k}$, representados como:
\begin{equation}
	\vec{i} =
	\begin{bmatrix}
		1 \\ 0 \\ 0
	\end{bmatrix},
	\vec{j} =
	\begin{bmatrix}
		0 \\ 1 \\ 0
	\end{bmatrix},
	\vec{k} =
	\begin{bmatrix}
		0 \\ 0 \\ 1
	\end{bmatrix}
\end{equation}

Entonces las matrices asimetricas $S(\vec{i}), S(\vec{j}), S(\vec{k})$ se definen como:

\begin{equation}
	S(\vec{i}) =
	\begin{bmatrix}
		0 & 0 & 0 \\
		0 & 0 & -1\\
		0 & 1 & 0
	\end{bmatrix},
	S(\vec{j}) =
	\begin{bmatrix}
		0 & 0 & 1 \\
		0 & 0 & 0\\
		-1& 0 & 0
	\end{bmatrix},
	S(\vec{k}) =
	\begin{bmatrix}
		0 & -1 & 0 \\
		1 & 0 & 0\\
		0 & 0 & 0
	\end{bmatrix}
\end{equation}


\end{document}
